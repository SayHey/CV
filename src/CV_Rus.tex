\documentclass[11pt,a4paper]{moderncv}
\moderncvtheme[blue]{classic}
\usepackage[utf8]{inputenc}
\usepackage[T2A]{fontenc}
\usepackage[english,russian]{babel}

\usepackage{relsize}
\usepackage{xcolor}
\usepackage[a4paper,margin=1in]{geometry}

\newcommand{\Csharp}{
  {\settoheight{\dimen0}{C}C\kern-.05em \resizebox{!}{\dimen0}{\raisebox{\depth}{\#}}}}
\newcommand\Cpp{C\nolinebreak[4]\hspace{-.05em}\raisebox{.4ex}{\relsize{-3}{\textbf{++}}}}

% Шапка
\firstname{Данил}
\familyname{Неверов}
% \title{сюда можно вписать что-нибудь в качестве заголовка}
\address{Санкт - Петербург}{ул. Лени Голикова}
\mobile{+7~(921)~903~96~86}
\email{danil.nev@gmail.com}
\homepage{github.com/sayhey}
\photo[90pt]{../img/2.png}
%\extrainfo{}
%\quote{Инженер с большим опытом написания резюме в \LaTeX. Претендую на позицию
%инженера или старшего инженера-разработчика программного обеспечения}%

%\nopagenumbers{}

\renewcommand{\rmdefault}{cmr} % Шрифт с засечками
\renewcommand{\sfdefault}{cmss} % Шрифт без засечек
\renewcommand{\ttdefault}{cmtt} % Моноширинный шрифт

\begin{document}
\maketitle

% ОБРАЗОВАНИЕ --------------------------------------------------------
\section{Образование}

\cventry{2008\,--\,2013}
{Санкт-Петербургский Государственный Университет}
{\newline Факультет Прикладной Математики -- Процессов Управления}
{\newline Специалист в области Прикладная математика и информатика}
{\newline Кафедра компьютерного моделирования и многопроцессорных систем}
{Диплом с отличием}  % arguments 3 to 6 can be left empty

\cventry{2005\,--\,2008}
{Академическая Гимназия СПБГУ (ФМШ-интернат №45)}
{Физ-Мат}{}{}{}

% ДИПЛОМНАЯ РАБОТА ---------------------------------------------------
\section{Дипломная работа}
\cvline{Название}{\emph{Оптимальное стохастическое управление с использованием функциональных интегралов в задачах машинного обучения}}
\cvline{Научный руководитель}{доктор физико-математических наук, профессор Сергей Николаевич Андрианов}
\cvline{Описание}{\small Разработан и протестирован алгоритм машинного обучения с подкреплением основанный на математическом аппарате теории стохастического управления и функционального интеграла.}

% ТЕХНИЧЕСКИЕ НАВЫКИ -------------------------------------------------
\section{Технические навыки}
\cvline{Языки и инструменты}
{
  Профессиональная деятельность: \newline
  \textbf{\Cpp17, Visual Studio, Git,} базовые знания \textbf{G-Code} и \textbf{CAD/CAM} ПО;
  \newline
  Хобби и научная деятельность: \newline
  \textbf{Python, TensorFlow, Keras, OpenCV, Processing (Arduino)}.
}
\cvline{Навыки}
{
  Большой арсенал численных и аналитических методов прикладной математики. Опыт и навыки решения (в том числе творческого и нестандартного) сложных задач. Знания вычислительной и дифференциальной геометрии, методов оптимизации и теории управления, математического моделирования. Опыт промышленного программирования. Способность проектировать и конструировать комплексные системы.
}

% ИНОСТРАННЫЕ ЯЗЫКИ --------------------------------------------------
\section{Языки}
\cvline{Английский}{Full Professional working proficiency. Ежедневно использую в деловой переписке
                    и на совещаниях. Имею сертификат FCE на CEFR уровень C1.}
\cvline{Русский}{Родной.}


% ОПЫТ РАБОТЫ ---------------------------------------------------
\section{ОПЫТ РАБОТЫ}
\subsection{Cpp}
\cventry{2013--настоящее время}
{Разработчик программного обеспечения - Математик}
{\newline CIMCO Software}
{\newline Санкт-Петербург, Россия / Копенгаген, Дания}
{}
{Разработка математического ядра для CAD/CAM программного обеспечения.\newline{}
\begin{itemize}
\item Поддержка и расширение функционала 2D CAD редактора:
  \begin{itemize}%
  \item Создание геометрической библиотеки для работы со сплайнами;
  \item Создание геометрической библиотеки для работы с эллипсами.
  \end{itemize}
  \href{http://www.cimco.com/software/cimco-cnc-calc/strategies}
       {Информация: \underline{\emph{\textcolor[rgb]{0.00,0.00,0.50}{cimco.com/software/cimco-cnc-calc/strategies}}}}
\end{itemize}}

\vspace{20 pt}

\cventry{2014--настоящее время}
{Разработчик программного обеспечения - Математик}
{\newline RapidCam}
{\newline Санкт-Петербург, Россия / Копенгаген, Дания}
{}{Разработка математического ядра для CAD/CAM программного обеспечения.\newline{}
\begin{itemize}
\item Работа над крупным проектом создания ядра
      для автоматической генерации управляющих программ (траекторий)
      для ЧПУ станков (toolpath generation). В рамках проекта мной были
      реализованы несколько стратегий резки для 3-осевых фрезерных станков
      (3-axis milling strategies):
  \begin{itemize}%
  \item Parallel;
  \item Contour;
  \item Scallop/Constant Stepover. Данная стратегия является самой продвинутой среди 3d стратегий. В процессе создания необходимого функционала были разработаны и применены множество нетривиальных методов и алгоритмов, таких как
    \begin{itemize}
    \item триангулятор плоских многоугольников;
    \item построение многообразия, лежащего на заданном расстоянии от данного двухмерного многообразия (3d offset);
    \item качественная триангуляция этого многообразия;
    \item построение на данной триангуляции численного решения специального вида ДУ методом конечных элементов;
    \end{itemize}
  \item Абсолютно новые еще не имеющие коммерческих названий экспериментальные стратегии.
  \end{itemize}
  \href{http://www.mastercam.dk/hsm-performance-pack/product/machining-strategies}
       {Информация: \underline{\emph{\textcolor[rgb]{0.00,0.00,0.50}{mastercam.dk/hsm-performance-pack/product/machining-strategies}}}}
\end{itemize}}

\vspace{20 pt}

\cventry{2018}
{Хобби проект}
{\newline MiniFlow}
{}
{\newline \href{https://github.com/SayHey/MiniFlow}
               {GitHub репозиторий проекта : \underline{\emph{\textcolor[rgb]{0.00,0.00,0.50}{github.com/sayhey/miniflow}}}}}
{Личный самообразовательный проект, целью которого является создание фреймфорка для работы с вычислительными графами и решения задач построения и тренировки нейронной сети. Проект является упрощенной версией фреймворка TensorFlow и аналогов. Написан на \Cpp17 . В планах перевод библиотеки тензоров на CUDA.}

\newpage

% Олимпиады и Дополнительное образование--------------------------------
\section{Олимпиады и Дополнительное образование}
\cvlistitem{Многократный победитель городских олимпиад по физике и математике, олимпиад для школьников МатМеха и ФизФака СПБГУ, участник турнира Юных Физиков;}
\cvlistitem{Активный пользователь ресурсов онлайн обучения. Закончил Deep Learning Nanodegree и Self-Driving Car Engineer Nanodegree на Udacity. Прошел серию курсов Deep Learning Specialization на Coursera. Прослушал десятки других онлайн курсов.}
\cvlistitem{Прошел годовой курс по Data Mining и Машинному обучению от Data Mining Labs. В рамках курса участвовал в хакатонах и командных проектах по Data Mining'у, в том числе в мозговых штурмах задач с Kaggle.com}
\cvlistitem{Участник курсов и семинаров ПОМИ РАН по вычислительной и дифференциальной геометрии, топологии и математической физике;}
\cvlistitem{Участник семинаров по вычислительным аспектам высшей нервной деятельности Биологического факультета СПБГУ, участник летней школы "Белые ночи математических нейронаук";}


% Проекты --------------------------------------------------------------
\section{Хобби и Научные Проекты}
\cvlistitem{Являюсь энтузиастом робототехники, DIY культуры, Arduino и других микропроцессорных платформ. За плечами несколько хобби-проектов: два летающих робота коптера (большой трикоптер и маленький квадрокоптер), Led куб и др.; }
\cvlistitem{Интересуюсь искусственным интеллектом и машинным обучением. Реализовал много мелких мл-проектов от стандартных классификаторов картинок до таких проектов как: end-to-end convolutional deep learning агент для управления автомобилем по визуальной информации с камеры, deep reinforcement learning агент для управления квадрокоптером, система генерации лиц на GAN (generative adversarial networks), система генерации текстов на рекуррентных сетях, анализ временных рядов на LSTM.}
\cvlistitem{В рамках дипломного проекта был разработан алгоритм машинного обучения, основанный на сложном математическом аппарате теоретической физики. В данный момент работаю над переносом и адаптацией данного алгоритма (и более простых) на упомянутых выше роботов;}
\cvlistitem{Интерес к квантовым вычислениям вылился в успешно реализованный проект математического синтеза оптимального управления модели квантового регистра. Использовался сложный математический аппарат управления на группах Ли;}
\cvlistitem{Совместно с товарищами с биологического факультета занимался технической частью в проекте моделирования работы биологической нейронной сети.}

% Сферы научных и профессиональных интересов ---------------------------
\section{Сферы научных и профессиональных интересов}
\cvlistitem{Вычислительная и дифференциальная геометрия; Биоинформатика; Машинное обучение; Робототехника; Искусственный Интеллект; Квантовые вычисления; Математическое моделирование экономических и финансовых систем.}

\end{document}
